\documentclass{article}
\usepackage[utf8]{inputenc}
\usepackage[fleqn]{amsmath}
\usepackage{amssymb}
\usepackage{mathtools}
\usepackage{algpseudocode}
\usepackage{algorithm}
\title{Programozás Beadandó}
\author{Tompa Gábor }
\date{27 Március 2018}

\begin{document}

\maketitle

\section{Feladat}
Valósítsa meg az egész számokat tartalmazó zsák típust! Ábrázolja a zsák elemeit (az
előfordulás számukkal együtt) egy sorozatban! Implementálja a szokásos
műveleteket (elem betevése, kivétele, üres-e a halmaz, egy elem hányszor van a
zsákban), valamint két zsák szimmetrikus differenciáját (a közös elemek nem
kerülnek be a szimmetrikus differenciába), továbbá egy zsák kiírását!
\section{Zsak tipus}
A zsák típus (Vagy más neven multiset) a klasszikus halmaz kibővitese azzal az engedménnyel hogy egy elemet többször is tartalmazhat(aminek számosságát taroljuk).A megvalósításnak része meg a halmaz hossza is.
\section{Típus műveletek}
\subsection{Üres}
Egy metódus ami ellenőrzi hogy a megadott zsák üres e
\subsection{Szimmetrikus Differencia}

Eg metódus ami egy másik Zsák típust értekül kapva visszaadja a két Zsák szimmetrikus differenciáját(mindegyik elemet csak egyszer rak bele a differenciába)
\subsection{Betevés}
Egy metódus ami beletesz egy új elemet a Zsákba
\subsection{Kivetel}
Egy elemet kivesz a zsákból,de teljesen(ha többször van benne akkor mindegyiket kiveszi)
\subsection{Elem sokasága}
egy metódus ami ellenőrzi hogy az adott elem hányszor található meg a zsákban

\section{Reprezentáció}
A Zsákot egy kettő egész számokból álló struktúrából (ami az elemet és a számosságát tartalmazza)alkotott egy dimenziós vektor alkotja valamint a hossz változó ami szinten egy természetes szám,a könnyebb debugolhatosagert,es a nagy zsákok lehetőségéért.

\section{Implementáció}
\subsection{Szimmetrikus Differencia}
A szimmetrikus differencia O(n^2) alatti megvalósítása
\begin{algorithmic}[1]
        \Procedure{Szim\_Dif}{$other\_bag$}
            \State $temp\_container$
            \State $benne_van$
                \For{i in $this.tarolo\_meret$}
                \For{j in $other\_bag.tarolo\_meret$}
                    \If{$this.tarolo\_tomb[i].element$ = $other\_bag.tarolo\_tomb[j].element$}
                    \State true \leftarrow $benne\_van$
                    \EndIf
                    \EndFor
                    \If{$benne\_van$ = false}
                 \State    $this.tarolo\_tomb[i].element$ \leftarrow $PutIn(temp\_container$)
                    \EndIf
                \EndFor
        \EndProcedure
        \end{algorithmic}
        \subsection{Elem sokasaga}
        Egy elem sokaságának O(n) alatti megvalositasa
        \begin{algorithmic}[2]
        \Procedure{ElementCardinality}{$element$}
            \State 0 \leftarrow $cardinality$
                \For{$j$  $other\_bag.tarolo\_meret$}
                    \If{$this.tarolo\_tomb[i].element$ = $element$}
                    \State $cardinality$ \leftarrow $this.tarolo\_tomb[i].cardinality$
                    \EndIf
                \EndFor
        \EndProcedure
        \end{algorithmic}
\subsection{Betevés}
Beletesz egy elemet a zsákba,ugy hogy lefoglal egy eggyel nagyobb átmeneti tömböt,amibe áttölti az adatokat,és utána törli a régit,felvesz egy eggyel nagyobb tarolót,amibe az átmeneti tömbből áttölti az elemeket
\subsection{Kivetél}
Hasonló a betevés művelethez,csak kiveszi az elemet és az elem nélkül maradékot tölti fel
\subsection{Üres}
Igazat ad vissza ha a hossz adattag 0 hamisat ha nem(mivel csak a betevés metóduson keresztül veszünk fel elemeket ahol gondoskodunk a méret adattag növeléséről,igy mindig helyes érteket fog visszaadni)
\section{Osztaly}
A Zsák osztályt egy Bag osztály valósítja meg ami tartalmaz egy kételemű struktúrából álló dinamikus tömböt és egy hossz érteket,valamint definiálja a szimmetrikus differencia függvényt,a PutIn (betevésre szolgáló),deleteElement(kivételre szolgáló),Üres() függvényt,Egy kiíró WriteOut() metódust ,egy ElementCardinality() metódust ami visszaadja egy elem számosságát,egy segédfüggvényt ami visszaadja a taroló tömböt,valamint újradefiniálja az értékadó operatort,és tartalmaz másoló konstruktort az üres és az elemeket magadon kívül.

        \begin{tabular}{|r|l|}
            \hline
              \multicolumn{2}{c}{Bag} \\
            \hline
            - & tarolo\_meret : unsigned long long int\\
            - & tarolo\_tomb : element\_and\_cardinality[0\dots d]\\
            \hline
            + & Bag(int[],int length) \\
            + & Bag() \\
            + & Bag(Bag) \\
            + & Szim\_dif(Bag) : Bag\\
            + & PutIn(int) : Bag\\
            + & DeleteElement(int)\\
            + & WriteOut()\\
            + & ElementCardinality(int) : int\\
            + & operator =(Bag) : Bag\\
            \hline
        \end{tabular}
   
        A seged struktura:
       
       
       
                \begin{tabular}{|r|l|}
            \hline
              \multicolumn{2}{c}{element\_and\_cardinality} \\
            \hline
            - & element : int\\
            - & cardinality : int\\
            \hline
        \end{tabular}
\section{Tesztelési terv}
\begin{enumerate}
    \item Uj Zsákok létrehozása
    \begin{enumerate}
        \item Egy elemű zsák létrehozása
        \item Két elemű Zsák létrehozása
        \item Üres zsák létrehozása
    \end{enumerate}
    \item Üres Zsákba új elem
    \item 0 berakása a nullát mar tartalmazó zsákba
    \item több elemű zsák egyeretelmu maxelemmel,Elemszám ellenőrzéssel
    \item Sok betevési művelet
    \item Sok torlesi művelet
    \item Egymás után sokszor kivétel ugyanazzal az elemmel
    \item Egyenlő szimdif(szimdif funkcionálási tesztelése)
    \item Elem torlese meglevő elemekből
    \item Másoló konstruktor
    \item értékadás
    \item értékadás egyenlően
\end{enumerate}
\end{document}

